\documentclass[x11names]{report}
\usepackage[a4paper, total={6in, 9in}]{geometry}
\usepackage[skins]{tcolorbox}
\usepackage{tikz}
\usetikzlibrary{arrows}
\usetikzlibrary{calc}
\usepackage{pgfplots}
\pgfplotsset{compat=1.9}
\usepgflibrary{shapes.geometric}
\usepackage{xcolor}
\usepackage{amsmath}
%\usepackage{fouriernc}
\usepackage{mathrsfs}
\usepackage{amssymb}
\usepackage{hyperref}

%% custom
\renewcommand*\contentsname{Indice}
\setcounter{tocdepth}{4}
\setcounter{secnumdepth}{2}
\pgfplotsset{compat=1.15}

\usepackage[pagestyles]{titlesec}
\titleformat{\chapter}[display]{\normalfont\bfseries}{}{0pt}{\Huge}
\newpagestyle{mystyle}
{\sethead[\thepage][][\chaptertitle]{}{}{\thepage}}
\pagestyle{mystyle}


% boxes
\definecolor{myblue}{RGB}{224, 245, 255} 
\definecolor{myred}{RGB}{198, 247, 211} 
\definecolor{myorange}{RGB}{255, 102, 0} 
\definecolor{mypurple}{RGB}{255, 102, 0} 

\newtcolorbox{es}[2][]{%
	enhanced,sharp corners,boxrule=0.4pt,colback=white, colframe=black, coltitle=black,fonttitle=\itshape, 
	attach boxed title to top left={yshift=-0.5\baselineskip-0.4pt,xshift=2mm},
	boxed title style={tile, size=minimal, left=0.5mm, right=0.5mm,
		colback=white, before upper=\strut, boxrule=0.5pt, colframe=black},
	title=#2,top=1em,#1 
}
\newtcolorbox{dym}[2][]{%
	enhanced,colback=white,colframe=black,coltitle=black,
	sharp corners,boxrule=0pt, %0.4pt
	fonttitle=\itshape,,
	attach boxed title to top left={yshift=-0.5\baselineskip-0.4pt,xshift=2mm},
	boxed title style={tile,size=minimal,left=2.5mm,right=0.5mm,
		colback=white,before upper=\strut},
	title=#2,top=1em,#1
}
\newtcolorbox{blues}[2][]{%
	enhanced,colback=myblue,colframe=black,coltitle=black,
	sharp corners,boxrule=0.4pt,
	fonttitle=\bfseries\itshape,
	attach boxed title to top left={yshift=-0.5\baselineskip-0.4pt,xshift=2mm},
	boxed title style={tile,size=minimal,left=0.5mm,right=0.5mm,
		colback=myblue,before upper=\strut},
	title=#2,top=1em,#1
}
\newtcolorbox{redes}[2][]{%
	enhanced,colback=myred,colframe=black,coltitle=black,
	sharp corners,boxrule=0.4pt,
	fonttitle=\bfseries\itshape,
	attach boxed title to top left={yshift=-0.5\baselineskip-0.4pt,xshift=2mm},
	boxed title style={tile,size=minimal,left=0.5mm,right=0.5mm,
		colback=myred,before upper=\strut},
	title=#2,top=1em,#1
}
\newtcolorbox{coroll}[2][]{%
	enhanced, colback=white, colframe=black, coltitle=black,
	sharp corners, boxrule=0pt,
	fonttitle=\bfseries\itshape,
	attach boxed title to top left={yshift=-0.5\baselineskip-0.4pt,xshift=2mm},
	boxed title style={tile,size=minimal,
		left=1mm, right=1mm, % Horizontal padding
		top=1mm, bottom=1mm, % Vertical padding
		colback=myred,
		before upper=\strut},
	title=#2,top=1em,#1
}

\newcommand*{\QEDA}{\null\nobreak\hfill\ensuremath{\blacksquare}}%
\newcommand*{\QEDB}{\null\nobreak\hfill\ensuremath{\square}}%

\newcommand{\esempio}[2]{
	\begin{es}{esempio #1}
		#2
	\end{es}
}
\newcommand{\definizione}[2]{
	\begin{center}
		\fboxsep11pt
		\colorbox{myblue}{\begin{minipage}{5.75in}
				\begin{blues}{Definizione: #1}
					#2
				\end{blues}
		\end{minipage}}
	\end{center}
}
\newcommand{\teorema}[2]{
	\begin{center}
		\fboxsep11pt
		\colorbox{myred}{\begin{minipage}{5.75in}
				\begin{redes}{#1}
					#2
				\end{redes}
		\end{minipage}}
	\end{center}
}
\newcommand{\dimostrazione}[2]{
	\begin{dym}{dimostrazione#1}
		#2
		\QEDB
	\end{dym}
}
\newcommand{\corollario}[2]{
	\begin{center}
		\begin{coroll}{corollario#1}
			#2
		\end{coroll}
	\end{center}
}

\newcommand{\verde}[1]{\colorbox{myred}{$\displaystyle #1$}}

\pgfplotsset{my axis/.append style={height=9cm,width=9cm,grid=major,samples=100,yticklabel=\empty,xticklabel=\empty}}
\pgfplotsset{my plot/.append style={thick,samples=500}}

\begin{document}
	
\begin{titlepage}
   \begin{center}
       \vspace*{1cm}
        
       \textbf{\LARGE Relazione di laboratorio - Esperienza di Poisson}
       
       \vspace{0.3cm}
       \large \textit{Rate di una sorgente radioattiva} \\
       
       \vspace{0.5cm}
       \Large Federico Cesari \\
       
       \small 1096759

			
		\vspace{1cm}
		\begin{center}
			\includegraphics[scale=1.2]{geiger.jpeg}	
		\end{center}
		
		

       \vfill
            
       
            
       \vspace{0.8cm}
     
       
            
       corso A\\
       Università degli studi di Torino, Torino\\
       3 marzo 2024\\
       
            
   \end{center}
\end{titlepage}

\tableofcontents
\newpage
	
\chapter{Elettrostatica}
\section{Campo elettrostatico}
\definizione{Campo elettrostatico}{
Il campo elettrostatico prodotto in un punto \(P\) da un sistema di cariche ferme è definito come forza elettrostatica risultante \(F\) che agisce su una carica di prova \(q_0\) posta in \(P\) divisa per la carica stessa.
}
In realtà la carica di prova può perturbare la distribuzione di cariche originale, per questo la definizione diventerebbe più precisa se si facesse tendere a zero il valore di \(q_0\). In pratica è sufficiente che \(q_0\) sia molto piccola rispetto alle altre cariche.
\subsection{Dipolo elettrico}
Definiamo \textbf{momento del dipolo}
\[
\vec{p} = q\vec{a}
\]
Consideriamo un punto \(P\) nel quale misuriamo il potenziale, pari alla somma dei contributi delle due cariche:
\[
V(P) = \frac{q}{4\pi \varepsilon_0} \left(\frac{1}{r_1} - \frac{1}{r_2}\right) = \frac{q}{4\pi \varepsilon_0} \left(\frac{r_2 - r_1}{r_1r_2}\right)
\]
Ipotizziamo che \(P\) sia molto distante dal dipolo (rispetto alla distanza \(a\)) così che \(r_1\) e \(r_2\) siano sempre più assimilabili a due segmenti paralleli. Andiamo poi a tracciare un segmento perpendicolare a \(r_1\) fino a \(r_2\) evidenziando la distanza \(r_2 - r_1 \approx a\cos\vartheta\). 

\subsubsection{Potenziale del dipolo}
Con le seguenti approssimazioni andiamo a riscrivere il potenziale:
\[
r_2 - r_1 \approx a\cos\vartheta \qquad \qquad r_1r_2 \approx r^2
\]
\[
V(P) = \frac{q}{4\pi \varepsilon_0} \left(\frac{a\cos\vartheta}{r^2}\right) = \frac{p}{4\pi \varepsilon_0 r^2}
\]
Vediamo come a grandi distanze il potenziale del punto \(P\) dia \textit{informazioni solo sul momento di dipolo} e non sulle due cariche o sulla loro distanza reciproca. A parità di momento si potranno avere due cariche ravvicinate e molto cariche o più distanti e meno cariche.


\subsubsection{Campo elettrico del dipolo}
Riscrivendo \(r\) e \(\cos\vartheta\) esprimiamo il potenziale come
\[
V(P) = \frac{q}{4\pi \varepsilon_0} \left(\frac{a\textcolor{orange}{\cos\vartheta}}{\textcolor{red}{r^2}}\right) 
\]
\[
V(P)= \frac{p}{4\pi\varepsilon_0}\frac{1}{\textcolor{red}{x^2 + y^2 + z^2}}\textcolor{orange}{\frac{z}{\sqrt{x^2 + y^2 + z^2}}} = \frac{p}{4\pi\varepsilon_0}\frac{z}{(x^2+y^2+z^2)^{\frac{3}{2}}}
\]\\

\noindent
Dalla relazione \(\vec{E} = -\vec{\nabla}V\) otteniamo l'espressioni del campo elettrico
\[
\vec{E} = \left( \frac{p}{4\pi\varepsilon_0}\frac{3xz}{r^5},\;\ \frac{p}{4\pi\varepsilon_0}\frac{3yz}{r^5},\;\ \frac{p}{4\pi\varepsilon_0}\left(\frac{3z^2}{r^5}-\frac{1}{r^3}\right)\right) 
\]

\[
E(P) = \frac{p}{4\pi\varepsilon_0r^3}\sqrt{3\cos^2\vartheta + 1}
\]
Concludiamo osservando che il campo elettrico del dipolo varia con \(r^3\) (il monopolo con \(r^2\)) e dipende anche dall'angolo \(\vartheta\).


\section{Lavoro elettrico}
Quando su una carica \(q_0\) agisce una forza \(\vec{F}\) \textit{di qualsiasi natura}, possiamo definire sempre un campo elettrico \(\vec{E}\) chi chiamiamo \textbf{campo elettromotore}
\[
\vec{E} = \frac{\vec{E}}{q_0} \;\ \to \;\ \vec{F} = q_0 \vec{E}
\]
di cui non conosciamo necessariamente la natura; nel caso di un insieme di cariche ferme questo coinciderà con il campo elettrostatico, ma in questo caso studiamo il caso più generale.

Suddividendo il percorso da \(A\) a \(B\) in segmenti infinitesimi \(d\vec{s}\) calcoliamo il lavoro come integrale di linea del campo \(E\) lungo \(C_1\) 

\[
dW_1 = \vec{F}\cdot d\vec{s} = q_0\vec{E} \cdot d\vec{s}
\]
\[
 W_1 =  q_0 \int_{C_1} \vec{E} \cdot d\vec{s}
\]

\subsection{Tensione elettrica e forza elettromotrice}
\definizione{Tensione elettrica}{
Il rapporto tra il lavoro compiuto dalla forza \(\vec{F}\) nello spostamento da \(A\) a \(B\) lungo il percorso \(C_1\) e la carica 

\[
T(A\to B,C_1) \;\ = \;\ \frac{W_1}{q_0} \;\ =\;\ \int_{C_1}\vec{E}\cdot d\vec{s}
\]
viene definito \textbf{tensione elettrica}.
}
Se si considera un secondo percorso \(C_2\) si trova in generale un lavoro diverso e quindi un diverso valore della tensione elettrica pur essendo i punti \(A\) e \(B\) gli stessi.


Se studiamo il lavoro lungo un percorso chiuso invece troviamo
\[
W = \oint \vec{F} \cdot d\vec{s} = q_0\oint\vec{E}\cdot d\vec{s}
\]
\definizione{Forza elettromotrice}{
Il rapporto tra lavoro compiuto sulla carica e la carica stessa per il percorso chiuso, ovvero l'integrale di linea circuitazione di campo elettrico
\[
\mathcal{E} = \oint\vec{E}\cdot d\vec{s}
\]
viene definito \textbf{forza elettromotrice} relativa al percorso chiuso \(C\).
}
Sviluppando l'integrale
\begin{align*}
	\oint\vec{F}\cdot d\vec{s} =& \int_{C_1}\vec{F}\cdot d\vec{s} + \int_{-C_2}\vec{F}\cdot d\vec{s} \\
							   =& \int_{C_1}\vec{F}\cdot d\vec{s} - \int_{C_2}\vec{F}\cdot d\vec{s} \\
							   =& W_1 - W_2
\end{align*}
vediamo che in generale per un percorso chiuso il lavoro è diverso da zero. Tuttavia andremo a studiare il comportamento del \textit{campo elettrostatico} che è conservativo e di conseguenza ha circuitazione nulla. 


\subsection{Potenziale elettrostatico ed energia potenziale}
Se il campo è conservativo, non dipendendo dal percorso  seguito l'espressione della tensione elettrica può essere riscritta semplicemente come l'integrale tra \(A\) e \(B\) 
\[
\int_{A}^B\vec{E}\cdot d\vec{s} = f(B) - f(A)
\]
dove \(f\) è una particolare funzione definita a meno di una costante arbitraria e che rinominiamo \(V\) \textbf{potenziale elettrostatico}. 
\definizione{Differenza di potenziale}{
Possiamo definire una \textbf{differenza di potenziale} pari a 
\[
\Delta V = - \int_{A}^B\vec{E}\cdot d\vec{s} 
\]
}



\section{Flusso di campo elettrico}
Definiamo flusso di un campo vettoriale \(\vec{E}\) l'integrale
\[
\Phi (\vec{E}) = \int_{\Sigma} \vec{E}\cdot \hat{u}_n \: d\Sigma
\]
dove \(\hat{u}_n\) è il versore normale alla porzione infinitesima di superficie \(d\Sigma\). Vediamo che il prodotto scalare fa sì che contribuisca solo la componente di campo vettoriale ortogonale alla superficie. 

Mostriamo come il flusso \textit{dipenda solo dall'angolo solido sotto il quale la superficie vede la carica}. Prima di tutto qualche osservazione geometrica:

\[
d\vartheta = \frac{ds}{r} \qquad \qquad ds' \cos\alpha = ds \;\ \to \;\ d\vartheta = \frac{ds'\cos\alpha}{r}
\]
\[
d\Omega= \frac{d\Sigma_0}{r^2} \qquad \qquad d\Sigma \cos\alpha = d\Sigma_0 \;\ \to \;\ d\Omega = \frac{d\Sigma\cos\alpha}{r^2}
\]
Andiamo ora a calcolare il flusso attraverso \(d\Sigma\):
\begin{align*}
	d\Phi =& \vec{E} \cdot \hat{u}_n d\Sigma \\ 
		  =& \frac{q}{4\pi\varepsilon_0 r^2} \hat{u}_r \cdot \hat{u}_n \: d\Sigma \\
		  =& \frac{q}{4\pi\varepsilon_0 r^2} |\hat{u}_r||\hat{u}_n|\textcolor{red}{\cos\alpha\: d\Sigma} \\ 
		  =& \frac{q}{4\pi\varepsilon_0 r^2} |\hat{u}_r||\hat{u}_n|\textcolor{red}{\: d\Sigma_0} \\
		  =& \frac{q}{4\pi\varepsilon_0  \textcolor{orange}{r^2}} \textcolor{orange}{\: d\Sigma_0} \\
		  =& \frac{q}{4\pi\varepsilon_0}\: \textcolor{orange}{d\Omega} \quad \to \quad \Phi = \frac{q}{4\pi\varepsilon_0}\Omega
\end{align*}
Quindi se consideriamo una superficie chiusa si ha un angolo solido 
\[
d\Omega = \frac{4\pi r^2}{r^2} = 4\pi
\]
e di conseguenza
\begin{equation}
	\verde{\Phi = \frac{q}{\varepsilon_0}}
\end{equation}
Se la carica esterna il flusso è nullo (le cariche esterne contribuiscono solo al campo elettrico). Inoltre se consideriamo più cariche, o addirittura una distribuzione omogenea di cariche si hanno i seguenti valori di flusso:
\[
\text{distribuzione finita di cariche: }\qquad \Phi = \oint \vec{E} \cdot \hat{u}_n \: d\Sigma \;\ = \;\ \sum_i \frac{q_{i(int)}}{\varepsilon_0}
\]
\[
\text{distribuzione continua di cariche: }\qquad \Phi = \frac{1}{\varepsilon_0}\int \rho(x,y,z) \:d\tau 
\]




\subsubsection*{Rotore di un campo vettoriale}
Viene definito rotore del campo elettrico \(\vec{E}\) il prodotto vettoriale
\[
\vec{\nabla} \wedge \vec{E} = \left|\begin{array}{ccc}
	\hat{u}_x & \hat{u}_y & \hat{u}_z \\
	\frac{\partial}{\partial x} & \frac{\partial}{\partial y} & \frac{\partial}{\partial z} \\
	E_x & E_y & E_z 
\end{array}\right| = \left(\frac{\partial E_z}{\partial y}-\frac{\partial E_y}{\partial z}\right)\hat{u}_x + \left(\frac{\partial E_x}{\partial z}-\frac{\partial E_z}{\partial x}\right)\hat{u}_y + \left(\frac{\partial E_y}{\partial x}-\frac{\partial E_x}{\partial y}\right)\hat{u}_z
\]
Questo rappresenta la capacità del campo elettrico di \textit{formare vortici}, ovvero di generare linee di forza che si richiudono su loro stesse.


Poiché anche il rotore del campo elettrico è un campo vettoriale, è possibile definirne un suo flusso:

\[
\Phi(\vec{\nabla} \wedge \vec{E}) =  \int_{\Sigma}\left(\vec{\nabla} \wedge \vec{E}\right)\cdot \hat{u}_n \: d\Sigma
\]

\teorema{}{
\subsubsection{Teorema di Stokes}
La circuitazione è uguale al flusso del rotore:
\[
\oint_\gamma \vec{E}\cdot d\vec{s} = \int_{\Sigma_\gamma}\left(\vec{\nabla} \wedge \vec{E}\right)\cdot \hat{u}_n \: d\Sigma
\]
Qui è espresso nel caso del campo elettrico dove si ha che la circuitazione è nulla qualunque sia la curva \(\gamma\) (a patto che sia chiusa). Allora il flusso del rotore è nullo qualunque sia la superficie ed è possibile solo se il rotore di \(\vec{E}\) è nullo:

\[
\vec{\nabla} \wedge \vec{E} = 0
\]

Dire che il rotore è nullo equivale a dire che il capo elettrico è \textbf{irrotazionale}, ovvero che le sue linee di forza non possono chiudersi su loro stesse; il campo elettrico "non forma vortici".
}

L'annullarsi del rotore non è un fatto sorprendente. Sappiamo infatti che il rotore di un gradiente è sempre nullo e che il campo elettrostatico conservativo può essere scritto come gradiente della funzione scalare del potenziale elettrostatico \(V\):
\[
\vec{\nabla} \wedge \vec{E} = \vec{\nabla} \wedge \left(-\vec{\nabla} V\right)
\]

\subsection{Discontinuità di carica}


\section{Conduttori}


\end{document}