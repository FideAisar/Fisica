\documentclass[x11names]{article}
\usepackage[a4paper, total={6in, 9in}]{geometry}
\usepackage[skins]{tcolorbox}
\usepackage{tikz}
\usetikzlibrary{arrows}
\usetikzlibrary{calc}
\usepackage{pgfplots}
\pgfplotsset{compat=1.9}
\usepgflibrary{shapes.geometric}
\usepackage{xcolor}
\usepackage{amsmath}
\usepackage{fouriernc}
\usepackage{mathrsfs}
\usepackage{amssymb}
\usepackage{hyperref}

%% custom
\renewcommand*\contentsname{Indice}
\setcounter{tocdepth}{4}
\setcounter{secnumdepth}{2}
\pgfplotsset{compat=1.15}

% boxes
\definecolor{myblue}{RGB}{224, 245, 255} 
\definecolor{myred}{RGB}{234, 222, 255}
\definecolor{myorange}{RGB}{255, 102, 0}

\newtcolorbox{es}[2][]{%
	enhanced,colback=white,colframe=black,coltitle=black,
	sharp corners,boxrule=0.4pt,
	fonttitle=\itshape,frame style={dash pattern=on 3pt off 3pt},
	attach boxed title to top left={yshift=-0.5\baselineskip-0.4pt,xshift=2mm},
	boxed title style={tile,size=minimal,left=0.5mm,right=0.5mm,
		colback=white,before upper=\strut},
	title=#2,#1
}
\newtcolorbox{dym}[2][]{%
	enhanced,colback=white,colframe=black,coltitle=black,
	sharp corners,boxrule=0.4pt,
	fonttitle=\itshape,,
	attach boxed title to top left={yshift=-0.5\baselineskip-0.4pt,xshift=2mm},
	boxed title style={tile,size=minimal,left=0.5mm,right=0.5mm,
		colback=white,before upper=\strut},
	title=#2,#1
}
\newtcolorbox{blues}[2][]{%
	enhanced,colback=myblue,colframe=black,coltitle=black,
	sharp corners,boxrule=0.4pt,
	attach boxed title to top left={yshift=-0.5\baselineskip-0.4pt,xshift=2mm},
	boxed title style={tile,size=minimal,left=0.5mm,right=0.5mm,
		colback=myblue,before upper=\strut},
	title=#2,#1
}
\newtcolorbox{redes}[2][]{%
	enhanced,colback=myred,colframe=black,coltitle=black,
	sharp corners,boxrule=0.4pt,
	fonttitle=\itshape,
	attach boxed title to top left={yshift=-0.5\baselineskip-0.4pt,xshift=2mm},
	boxed title style={tile,size=minimal,left=0.5mm,right=0.5mm,
		colback=myred,before upper=\strut},
	title=#2,#1
}


\newcommand*{\QEDA}{\null\nobreak\hfill\ensuremath{\blacksquare}}%
\newcommand*{\QEDB}{\null\nobreak\hfill\ensuremath{\square}}%

\newcommand{\esempio}[2]{
	\begin{es}{esempio #1}
		#2
	\end{es}
}
\newcommand{\definizione}[2]{
	\begin{center}
		\fboxsep11pt
		\colorbox{myblue}{\begin{minipage}{5.75in}
				\begin{blues}{Definizione: #1}
					#2
				\end{blues}
		\end{minipage}}
	\end{center}
}
\newcommand{\teorema}[2]{
	\begin{center}
		\fboxsep11pt
		\colorbox{myred}{\begin{minipage}{5.75in}
				\begin{redes}{#1}
					#2
				\end{redes}
		\end{minipage}}
	\end{center}
}
\newcommand{\dimostrazione}[2]{
	\begin{dym}{dimostrazione: #1}
		#2
		\QEDB
	\end{dym}
}

\pgfplotsset{my axis/.append style={height=9cm,width=9cm,grid=major,samples=100,yticklabel=\empty,xticklabel=\empty}}
\pgfplotsset{my plot/.append style={thick,samples=500}}

\begin{document}
	
\begin{titlepage}
   \begin{center}
       \vspace*{1cm}
        
       \textbf{\LARGE Relazione di laboratorio - Esperienza di Poisson}
       
       \vspace{0.3cm}
       \large \textit{Rate di una sorgente radioattiva} \\
       
       \vspace{0.5cm}
       \Large Federico Cesari \\
       
       \small 1096759

			
		\vspace{1cm}
		\begin{center}
			\includegraphics[scale=1.2]{geiger.jpeg}	
		\end{center}
		
		

       \vfill
            
       
            
       \vspace{0.8cm}
     
       
            
       corso A\\
       Università degli studi di Torino, Torino\\
       3 marzo 2024\\
       
            
   \end{center}
\end{titlepage}

\tableofcontents
\newpage
	

\subsection{Campo Elettico}
\paragraph{Campo elettrico di un anello}
\paragraph{Campo elettrico di un disco}	
\paragraph{Campo elettrico di un piano}

\subsubsection{Linee di campo}

\subsection{Tensione, forza elettromotrice e energia potenziale}

\[ 
dW = \vec{F}\cdot d\vec{s} = q \vec{E} \cdot d\vec{s} = qE\cos\vartheta ds
\]
\[ 
W(C_{1}) = \int_{C_{1}} q\vec{E} \cdot d\vec{s}
\]
il valore
\[ 
\frac{W(C_{1})}{q} = \int_{C_{1}} \vec{E} \cdot d\vec{s}
\]
prende il nome di \textbf{tensione elettrica} tra \(A\) e \(B\) lungo la traiettoria \(C_{1}\).
\[ 
W = \oint q\vec{E}\cdot d\vec{s}
\]
Se il campo è conservativo possiamo dare la definizione di energia potenziale
\[ 
\Delta U = -W_{AB} = - \int_{A}^B q\vec{E}\cdot d\vec{s}
\]
e di differenza di potenziale
\[ 
\Delta V = \frac{\Delta U}{q} = - \int_{A}^B \vec{E}\cdot d\vec{s}
\]

\end{document}