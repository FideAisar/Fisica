\documentclass{article}
\usepackage{tikz}
\usepackage{pgfplots}
\usepackage{xcolor}
\usepackage{svg}
\usepackage{amsmath}
\usepackage{array}
\usepackage[skins]{tcolorbox}
\usepackage[version=4]{mhchem}
\usepackage[a4paper, total={6in, 8in}]{geometry}
\usepackage{fouriernc}
\usepackage{xymtex}
\usepackage{textcomp}
\usepackage{eurosym}
\usepackage{caption}
\usepackage{longtable}


\title{Relazione - pendolo di Kater}
\author{Federico Cesari}
\date{gennaio 2024}




\begin{document}
	\maketitle
	\vspace{3cm}
	
	L'esperienza di laboratorio ha lo scopo di determinare il periodo di oscillazione di un pendolo fisico in presenza di errori casuali (ed eventualmente di errori sistematici) e verificare che le misure osservate sono accurate o meno rispetto alle misure della fotocellula.
	
	
	\textit{indicare la strumentazione utilizzata con relativa sensibilità}
	
	
	\textit{descrivere brevemente la procedura di misura effettuata}
	
	
	
	
	
	
	
	
	
	\newpage
	\section{Punto 4}
	
	Dalle cento misure ho ricavato i valori nella prima tabella. Il più piccolo valore rilevato ($1.59s$) e il più grande ($2.31s$) decido di escluderli secondo il criterio di rigetto a $3\sigma$: so infatti che le mie misure hanno il $99.7 \%$ di probabilità di ricadere nell'intervallo $(\mu  - 3\sigma , \mu + 3\sigma)$, dove $\mu$ è la media della popolazione. Posso quindi affermare che valori osservati oltre $3 \sigma$ appartengono a un'altra popolazione e, di conseguenza, che è lecito rigettarli. \\ 
	
	\noindent
	Successivamente al rigetto, con i $98$ dati rimanenti ricavo i nuovi valori riportati nella seconda tabella.
	
	\vspace{0.8cm}
	\begin{minipage}[c]{0.45\textwidth}
		\centering
		\begin{tabular}{llrl}
			Media                       & $\bar{x}$             & $1.95$        & $s$       \\
			Varianza                    & $\sigma ^ 2$          & $0.0092$     & $s^2$  \\
			Dev. std                    & $\sigma$              & $0.096$      & $s$   \\
			Dev. std (della media)      & $\sigma_{\bar{x}}$    & $0.0096$     & $s$    \\
			Mediana                     & $\mu_e$               & $1.95$        &  $s$      \\
			Moda                        & $v_0$                 & $2.00$        & $s$
		\end{tabular}
		\captionof{table}{\textbf{100 misure}}
	\end{minipage}
	\begin{minipage}[c]{0.5\textwidth}
		\centering
		\begin{tabular}{llrl}
			Media                       & $\bar{x}$             & $1.95$        & $s$       \\
			Varianza                    & $\sigma ^ 2$          & $0.0068$     & $s^2$  \\
			Dev. std                    & $\sigma$              & $0.082$      & $s$   \\
			Dev. std (della media)      & $\sigma_{\bar{x}}$    & $0.0083$     & $s$    \\
			Mediana                     & $\mu_e$               & $1.95$        &  $s$      \\
			Moda                        & $v_0$                 & $2.00$        & $s$
		\end{tabular}
		\captionof{table}{\textbf{98 misure}}
	\end{minipage}
	\vspace{0.8cm}
	
	\noindent
	Confrontando i valori nelle due tabelle si notano una sensibile diminuzione della varianza ($-26 \%$) e una variazione della deviazione standard e della deviazione standard della media (entrambe circa $-13 \%$). Variazioni prevedibili vista l'esclusione di valori molto distanti dalla media della popolazione. 
	
	\vspace{0.8cm}
	\begin{table}[hb]
		\centering
		\begin{tabular}{llrl}
			Media                       & $\bar{x}$             & $1.95$        & $s$       \\
			Varianza                    & $\sigma ^ 2$          & $0.0067$     & $s^2$  \\
			Dev. std                    & $\sigma$              & $0.082$      & $s$   \\
			Dev. std (della media)      & $\sigma_{\bar{x}}$    & $0.0083$     & $s$    
		\end{tabular}
		\captionof{table}{\textbf{Dati accorpati}}
	\end{table}
	\vspace{0.5cm}
	
	\noindent
	I dati accorpati producono valori pressoché identici a quelli rilevati post rigetto a $3\sigma$. \\
	
	
	\noindent
	La sensibilità dello strumento è $0.01s$, ciò significa che lo strumento non può distinguere o rilevare variazioni inferiori a tale valore. La deviazione standard da me calcolata mi dice invece che le mie misure hanno una dispersione naturale attorno al valor medio uguale a $0.008s$, quindi minore della sensibilità dello strumento. Scelgo $0.01s$ come errore sulla stima, essendo questo il più piccolo valore misurabile. \\
	
	
	\noindent
	Dopo aver escluso i dati appartenenti a un'altra popolazione e raggruppato i dati, la mia migliore stima del  periodo di oscillazione del pendolo è:
	\[
	1.95s \quad \pm \quad 0.01 s
	\]
	
	
	
	
	
	\newpage
	\section{Test del $\chi ^2$}
	
	Sappiamo che una distribuzione di Gauss centrata in $\mu = \bar{x}$ è la distribuzione limite che dovrebbe descrivere meglio l'istogramma sperimentale.
	
	
	
	\newpage
	\section{Punto 6}
	
	\newpage
	\section{Punto 7}
	\newpage
	\section{Punto 8}
	
	\newpage
	\section{Punto 9}
	
	
	
\end{document}
