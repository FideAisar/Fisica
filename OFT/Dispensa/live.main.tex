undefined
\begin{document}

\tableofcontents

\chapter{Termodinamica - I principio}
\begin{es}{}
	\begin{enumerate}
		\item \textbf{Temperatura} come indice di equilibrio termico tra sistemi;
		\item \textbf{Primo principio} della termodinamica come equazione fondamentale dei bilanci energetici;
		\item Il \textbf{calore} come scambio energetico.
	\end{enumerate}
\end{es}
	

\textcolor{red}{Inserire piccolo riassunto pagina 426}\\

\textcolor{red}{Lorem ipsum dolor sit amet, consectetur adipiscing elit, sed do eiusmod tempor incididunt ut labore et dolore magna aliqua. Ut enim ad minim veniam, quis nostrud exercitation ullamco laboris nisi ut aliquip ex ea commodo consequat. Duis aute irure dolor in reprehenderit in voluptate velit esse cillum dolore eu fugiat nulla pariatur. Excepteur sint occaecat cupidatat non proident, sunt in culpa qui officia deserunt mollit anim id est laborum}


\section{Termometria}
Chiamiamo \textbf{sistema termodinamico} una porzione del mondo costituita da una oo più parti assimilabile a un sistema continuo macroscopicamente costituito da un numero di elementi pari a \(N_A = 6.0221 \cdot 10^{23}\). L'oggetto di studio saranno le proprietà fisiche macroscopiche e le loro varzioni, \textbf{trasformazioni} del sistema e \textbf{scambi energetici}.

Quando si parlerà di \textbf{ambiente} si intenderà un insieme di una o più parti con cui il sistema termodinamico può interagire. L'ambiente quindi \textbf{contribuisce} in generale \textbf{a determinare le caratteristiche fisiche mascroscopiche del sistema}.

Distinguiamo:
\begin{enumerate}
	\item \textbf{Sistma aperto} se avvengono scambi di energia e materia con l'ambiente;
	\item \textbf{Sistma chiuso} se avvengono solo scambi di energia con l'ambiente;
	\item \textbf{Sistma isolato} se non avvengono scambi con l'ambiente.
\end{enumerate}

\[ 
\viola{\text{{L'insieme (sistema + ambiente) è detto \textbf{universo termodinamico}}}}
\]
\subsection{Variabili termodinamiche}
Al fine di descrivere il nostro sistema termodinamico viene utilizzato un numero ridotto di variabili termodinamiche, ovvero grandezze fisiche \textbf{direttamente misurabili}. Le distinguiamo in estensive ed intensive:

\begin{center}
	\begin{minipage}{0.49\textwidth}
		\begin{es}{Estensive}
			Variabili \textit{additive} che esprimono proprità \textbf{globali} del sistema che dpendono in particolare dalle dimensioni o dall'estensione di quest'ultimo.
			
			\[\viola{\text{Massa}} \qquad \viola{\text{Volume}}\]
			
		\end{es}
	\end{minipage}
	\begin{minipage}{0.49\textwidth}
		\begin{es}{Intensive}
			Variabili \textit{non additive} che esprimono proprità \textbf{locali} del sistema che possono variare da punto a punto.
			
			\[\viola{\text{Pressione}} \quad \viola{\text{Temperatura}}\quad \viola{\text{Pressione}}\]
			
		\end{es}
	\end{minipage}
\end{center}
In base al sistema termodinamico in esame saranno necessarie certe variabili termodinamiche; per esempio per descrivere lo stato di un gas ideale ne sono necessarie 3 \(pVT\) poiché legate dall'equazione di stato \( pV = nRT\) in cui due sono variabili e una dipendente dalle 2.\\

\noindent
\textbf{Osservazione}  La definizione si stato termodinamico è concettualmente diversa da stato meccanico. Uno stato meccanico presuppone la conoscenza di posizione e velocità degli \(n\) corpi interessati, lo stato termodinamico invece, data l'elevato numero di elementi \(n\) non permette ciò, anzi, ad uno stato termodinamico possono essere associti più stati meccanici.

\section{Equilibrio termico}
\textcolor{red}{Lorem ipsum dolor sit amet, consectetur adipiscing elit, sed do eiusmod tempor incididunt ut labore et dolore magna aliqua. Ut enim ad minim veniam, quis nostrud exercitation ullamco laboris nisi ut aliquip ex ea commodo consequat. Duis aute irure dolor in reprehenderit in voluptate velit esse cillum dolore eu fugiat nulla pariatur. Excepteur sint occaecat cupidatat non proident, sunt in culpa qui officia deserunt mollit anim id est laborum}\\

\noindent
Ora precisiamo il concetto di equilibrio termico. Presi due sistemi A e B con rispettive temperature \(T_A\) e \(T_B\), i sistemi si dicono in \textbf{equilibrio termico} se hanno la stessa temperatura, \(T_A = T_B\)
\begin{center}
	\fboxsep11pt
	\colorbox{yblue}{\begin{minipage}{5.75in}
			\begin{blues}{Temperatura}
				La temperatura è l'indice dell'equilibrio termico tra due sistemi.
			\end{blues}
	\end{minipage}}
\end{center}

Per portare due sistemi all'equilibrio termico bisogna porre questi in \textbf{contatto termico} tramite una parete. Se questa parete porta i due sistemi in equilibrio termico, allora prende il nome di  \textbf{parete diatermica}, se no \textbf{parete adiabatica}. \textit{Nella pratica la parete adiabatica e un caso ideale limite}. 

\begin{center}
	\fboxsep11pt
	\colorbox{yblue}{\begin{minipage}{5.75in}
			\begin{blues}{Sistema adiabatico}
				Un sistema è definito adiabatico se è circondato da pareti adiabatiche. Un sistema adiabatico non può essere messo in contatto termico con un'altro sistema o con l'ambiente. \textbf{Una parete è sempre necessaria} per evitare il contatto termico.
			\end{blues}
	\end{minipage}}
\end{center}

\textit{In generale per il contatto termico non per forza deve entrare in gioco una parete; due solidi a contatto o due fluidi immiscibili non ne hanno bisogno; la parete è necessaria in casi come il contenimento di un gas.}

\begin{center}
	\fboxsep11pt
	\colorbox{yblue}{\begin{minipage}{5.75in}
			\begin{blues}{Definizione operativa di temperatura}
				Per prima cosa occorre identificare un fenomeno fisico che dipenda dalla temperatura \(\theta\) e una grandezza \(X\) che lo caratterizzi. \(X\) è detta \textbf{caratteristica termometrica} e la temperatura funzione di \(X\) è detta \textbf{funzione termometrica} \(\theta(X)\).
				
				Il dispositivo in cui avviene il fenomeno e che fornisce il valore della caratteristica termometrica è indicato come \textbf{termometro}.
			\end{blues}
	\end{minipage}}
\end{center}

\begin{center}
	\begin{table}[H]
		\begin{tabular}{@{}lll@{}}
			\textbf{Termometro}   & \textbf{Fenomeno}   & \textbf{Caratteristica termometrica} \(X\)     \\
			\textit{a liquido}  & dilatazione termica di un liquido   & lunghezza della colonna di liquido     \\
			\textit{a resistenza}   & variazione della resistenza elettrica  & resistenza elettrica     \\
			\textit{a gas a volume costante}   & variazione della pressione  & pressione  \\        
		\end{tabular}
	\end{table}
\end{center}


Nonostante i nostri termometri mostrano una variazione di temperatura lineare, la dipendenza \(\theta(X)\) non lo è (talvolta non è neanche monotona). Nella pratica i termometri vengono utilizzati in intervalli di temperatura nei quali la dipendenza può essere \textbf{approssimata con un andamento lineare} \\
\[\theta(X) = aX \qquad a \text{ costante}\]\\
E' essenziale che il sistema di cui noi andiamo a misurare la temperatura sia riproducibile con facilità, per questo si definisce un \textbf{punto fisso}, un valore arbitrario associato al sistema in condizioni di equilibrio facilmente riproducibili. Il punto fisso campione scelto è il \textbf{punto triplo dell'acqua}: stato in cui ghiaccio e acqua e vapor d'acqua saturo sono in equilibrio; a questo stato è stata assegnata la temperatura arbitraria di 
\[ 
T_{pt} = \SI{273.16}{\kelvin}
\]
Da questa possiamo ricavare la \textbf{temperatura empirica} di un termometro:
\begin{enumerate}
	\item Si tara il termometro mettendolo in contatto termico con una cella al punto triplo dell'acqua; il termometro, raggiunto l'equilibrio, darà il valore \(X_{pt}\)
	\[ 
	\theta(X_{pt}) = aX_{pt} =  \SI{273.16}{}
	\]
	\item Si pone poi il termometro a contatto con il sistema a temperatura incognita; all'equilibrio il termometro fornirà il valore \(X\)
	\[ 
	\theta(X) = aX = \textcolor{red}{aX_{pt}}\frac{X}{X_{pt}} = \textcolor{red}{\theta(X_{pt})}\frac{X}{X_{pt}} = \textcolor{red}{273.16}\frac{X}{X_{pt}} 
	\]\\
	\begin{equation}\label{temperatura punto triplo}
		\viola{\theta(X) = T  = 273.16\frac{X}{X_{pt}} \SI{ }{[\kelvin]} }
	\end{equation}
	\\
	
	\noindent
	Da cui troviamo anche che la costante \(a\) vale \(a = 273.16/X_{pt}\).
	
	Termometri diversi forniscono sempre letture diverse qando in equilibrio termico con lo stesso stato di un sistema anche se tarati per costruzione al punto triplo dell'acqua (\textit{mio guess: cambierà il tipo di andamento, forse non più lineare}).
	
\end{enumerate}
\subsection{Termometro a gas}
\begin{center}
	\begin{minipage}{0.49\textwidth}
		\begin{center}
			\textcolor{red}{inserire grafici}
		\end{center}
	\end{minipage}
	\begin{minipage}{0.49\textwidth}
		\begin{center}
			\textcolor{red}{inserire grafici}
		\end{center}
	\end{minipage}
\end{center}
Si usa come caratteristica termometrica la pressione:

come prima si misurerà la pressione al punto triplo
\[ 
T_{pt} = Cp_{pt}
\]
dalla quale ricaviamo la temperatura \colorbox{attenzione}{(\textit{non è una definizione accurata})}
\[ 
T = T_{pt}\frac{p}{p_{pt}} = 273.16\frac{p}{p_{pt}}\SI{}{\kelvin}
\]
Come si può vedere dal grafico diminuendo la pressione nel bulbo le temperature dei gas tendono ad assumere valori uguali. Quindi posso dire che la \textbf{pressione} è una \textbf{buona caratteristica temometrica} solo quando è \textbf{bassa}. Quindi il termometro a gas è un buono strumento solo quando il gas è in \textbf{condizioni ideali}.

\[ 
T = T_{pt}\frac{p}{p_{pt}} = \left(\SI{273.16}{\kelvin}\right)\lim_{p\to 0}\left(\frac{p}{p_{pt}}\right)
\]

\subsection{Dilatazione termica}
Il volume di un corpo, a pressione costante, aumenta al crescere della temperatura. Presa una sbarra di lunghezza \(L\) a temperatura \(T\) dopo una variazione di temperatura \(\Delta T\) si ha
\[ 
L +\Delta L = L + \alpha L\Delta T
\]
\[ 
\viola{\frac{\Delta L}{L} = \alpha \Delta T}
\]
con \(\alpha\) coefficiente di dilatazione lineare medio della sostanza. Nel caso di un oggetto isotropo, per la dilatazine volumica si ha
\[ 
V + \Delta V = \left(L_1L_2L_3\right) + \left(\alpha L_1 \Delta T\right)\left(\alpha L_2 \Delta T\right)\left(\alpha L_3 \Delta T\right) = V + V(\alpha \Delta T)^3 \approx V + V(3\alpha \Delta T)
\]
\[ 
\viola{\frac{\Delta V}{V} = 3\alpha \Delta T}
\]


\section{Esperimenti di Joule}
Joule condusse una serie di esperimenti sugli\textbf{ effetti termici del lavoro meccanico} (attraverso un mulinello, una resistenza, un pistone, dei blocchi strofinati). Nelle varie esperienza il sistema costituito dall'acqua e dal dispositivo meccanico o elettrico è racchiuso entro \textbf{pareti adiabatiche}. Il risultato ottenuto è che 

\begin{center}
	il \textbf{lavoro}  speso a parità di massa d'acqua è sempre \textbf{proporzionale alla variazione di temperatura} dell'acqua con la stessa costante di proporzionalità.
	\[ 
	W_{ad} = C\Delta T
	\]
\end{center}
In analogi acon la definizione di energia potenziale per le forze conservative, introduciamo il concetto di \textbf{energia interna}, energia che dipende solo dallo stato del sistema (cioè dalle sue coordinate termodinamiche):
\[ 
W_{ad} = -\Delta U 
\]
\begin{es}{convenzione}
	Se il sistema \textbf{fornisce lavoro all'esterno} il lavoro è assunto \textbf{positivo}, se il sistema \textbf{riceve lavoro} dall'ambiente allora il lavoro è assunto \textbf{negativo}.
\end{es}
Lo stesso incremento di temperatura è ottenibile senza compiere lavoro meccanico mediante uno \textbf{scambio di calore}, per esempio immergendo un corpo più caldo. Possiamo ottenere lo stesso cambiamento dello stato termodinamico dell'acqua, segnalato dalla stessa variazaione di temparatura per il quale il sistema va ad aumentare la sua energia interna secondo
\[ 
Q = \Delta U
\]
dove \(Q\) rappresenta il calore scambiato senza lavoro esterno. Da questa otteniamo che 
\[ 
Q = \Delta U = -W
\]
\begin{equation}\label{Mayer}
	Q = -W_{ad}
\end{equation}
Esiste quindi un particolare scambio di energia che non comporta movimenti macroscopici al quale si dà il nome di \textbf{scambio di calore}.


\begin{center}
	\fboxsep11pt
	\colorbox{yred}{\begin{minipage}{5.75in}
			\begin{redes}{}
				\subsection{Primo principio della termodinamica}
				Se il sistema compie una trasformazione da uno stato A ad uno stato B, scambiando \textbf{calore} e \textbf{lavoro} con l'ambiente, \(Q\) e \(W\) \textbf{dipendono dalla particolare trasformazione} che congiunge i due stati termodinamici mentre \textbf{la differenza \(Q-W\) risulta indipendente} dalla trasformazione.\\
				
				Si può quindi scrivere che la differenza di energia interna 
				\[ 
				U_B - U_A = \Delta U 
				\]
				è uguale alla differenza \(Q-W\):
				\begin{equation}\label{primoprincipio}
					\Delta U = Q -W
				\end{equation}
			\end{redes}
	\end{minipage}}
\end{center}
\begin{center}
	\textcolor{red}{inserire grafici}
\end{center}

\subsection{Energia interna}
Esiste quindi una funzione di stato \textbf{energia interna} le cui variazioni misurano gli scambi di energia con l'esterno durante una trasformazione. Quando si fornisce energia a un sistema questa resta immagazzinata sotto forma di energia interna e può essere poi riutilizzata.

Da notare che l'energia interna non indica l'energia cinetica del sistema, o la sua energia potenziale, bensì energià legata a \textbf{proprietà interne del sistema} come moto molecolare o forze intermolecolari, infatti lo scambio di quest'energia  può avvenire non solo tramite lavoro, ma anche sotto forma di \textbf{scambio di calore}, riconducibile a fenomeni meccanici microscopici.\\

\noindent
Spesso è utile considerare trasformazioni con variazioni infinitesime del tipo
\[ 
dQ = dU + dW
\]
dove però \(\boldsymbol{dU}\) è un \textbf{differenziale esatto} perché come abbiamo detto prima la differenza di energia interna è indipendente dalla trasformazione, e \(\boldsymbol{dQ}\) e \(\boldsymbol{dW}\) non sono differenziali esatti poiché questi dipendono da \textit{come si è svolta la trasformazione} e quindi non posso essere espressi come differenza dei valori di una funzione di stato.

\subsection{Trasformazioni termodinamiche}
I valori di calore e lavoro \(Q\) e \(W\) possono essere calcolati direttamente solo in casi specifici se si conoscono le loro espressioni analitiche dal momento che cambiano con la trasformazione.  Se conosciamo le espressioni di \(\Delta U,Q,W\) in funzione delle coordiante termodinamiche 
\[ 
\Delta U = Q -W
\]
diventa un'eqwuazione che lega le coordinate termodinamiche durante la trasformazione, quindi diventa \textbf{l'equazione della trasformazione}.\\

\noindent
Prima di distinguere tre tipi di trasformazioni facciamo qualche osservazione.
\begin{es}{1. corpi a contatto}
	Prendiamo due corpi a temperature \(T_1\) e \(T_2\) i poniamoli in contatto termico in un contenitore adiabatico. Tra di essi avviene uno scambio di calore fino a che raggiungono l'equilibrio termico. Durante il processo c'è sempre una differenza di temperatura finita, quindi durante la trasformazione \textbf{non c'è mai equilibrio termico}.
\end{es}
\begin{es}{2. attrito}
	Preso un cormpo con velocità inizial \(v\) frenato da una forza d'attrito. L'energia cinetica diminuisce e contemporaneamente la temperatura delle superfici a contatto, del corpo e del piano, aumenta. Anche se assumessimo che questo processo avvenga in un tempo molto breve così da poter essere pensato adiabatico, alla fine i corpi cederebbero calore all'ambiente raggiungendo l'equilibrio termico. \\
	
	Quindi nella prima fase non c'è equilibrio meccanico: \(W = -\Delta U = \Delta K\), l'energia cinetica decresce quindi l'energia interna cresce. Nella seconda fase invece non c'è equilibrio termico: \(\Delta U = Q\) dove \(Q\) è ceduto all'ambiente quindi \(U\) decresce. Durante il processo quindi \textbf{tutti gli stati intermedi sono di non equilibrio}.
\end{es}


\begin{enumerate}
	\item \textbf{Trasformazioni adiabatiche}: una qualsiasi trasformazione in cui \(Q = 0\) (si ottiene per trasformazioni veloci) e quindi \(\Delta U = - W\). Il sistema non scambia calore con l'ambiente, ossia è \textbf{isolato termicamente}. Sperimentalmente si ottengono le condizioni adiabatiche chiudendo il sistema in un contenitore con pareti adiabatiche. Proprio per questo isolamento il sistema \textbf{non può raggiungere l'equilibrio termico}.
	\item \textbf{Trasformazioni reversibili}: avvengono attraverso stati di eqwuilibrio e in assenza di qualsiasi forza dissipativa. Sono utili perché possono essere arrestate in qualunque stato intermedio e se ne può invertire il verso variando di poco le condizioni esterne.
	\item \textbf{Trasformazioni irreversibili}: avvengono attraverso stati di non equilibrio e/o in presenza di forze dissipative.
\end{enumerate}

\begin{es}{3. equilibrio}
	Prendiamo una vasca d'acqua con all'interno un contenitore di gas. Le pareti del contenitore sono diatermiche quindi il gas è in equilibrio termico a temperatura costante \(T\). Si può espandere lentamente il contenitore muovendo una parete con una forza che sia sempre uguale e contraria a quella di pressione così da ottenere anche l'equilibrio meccanico. \\
	
	In questo caso tutti gli stati intermedi si possono considerare di equilibrio. Ciò può avvenire solo se si opera una trasformazione \textbf{quasi-statica}, ovvero: prima di procedere a una trasformazione infinitesima di stato si attende il ristabilirsi dell'equilibrio nella nuova condizione.
	
	\begin{center}
		\fboxsep11pt
		\colorbox{attenzione}{\begin{minipage}{5in}
				\begin{attenzione}{!!!}
					La lentezza delle trasformazioni è \textbf{condizione solo necessaria} e non sufficiente. Vediamo come nel caso 1 si otterrebbero sempre stati intermedi di non equilibrio.
				\end{attenzione}
		\end{minipage}}
	\end{center}
\end{es}


\section{Calorimetria}
Abbiamo visto dal primo teorema della termodinamiche che lo scambio di calore \(Q\) comporta una variazione di energia interna \(\Delta U\) e uno scambio di lavoro \(W\) secondo la legge
\[ 
\Delta U = Q -W
\]
Per semplicità andremo a studiare scambi di calore tra corpi solidi o liquidi andando a trascurare eventuali dilatazioni e di conseguenza il lavoro meccanico. \\

\noindent
In un contenitore adiabatico poniamo due corpi a temperature \(T_1\) e \(T_2\) a contatto  in modo che cedano calore \(Q\) fino ad arrivare all'equilibrio termico \(T_e\). Ognuno dei due corpi avrà una variazione della temperatura \(T_{e} - T_i\). 
Il sistema non scambia lavoro o calore con l'ambiente pertanto l'energia interna resta costante. Andando ad analizzare le energie interne ai due corpi, dovendo essere \(\Delta U = \Delta U_1 + \Delta U_2 = 0\) abbiamo \(\Delta U_1 = -\Delta U_2\); inoltre poiché non è stato compiuto un lavoro meccanico varrà \(Q_1 = -Q_2\): \textbf{il calore ceduto dal corpo è uguale in modulo a quello assorbito dall'acqua}.


Dalle misure si trova che esiste (nel limite di piccole variazioni di temperatura) \textbf{proporzionalità} tra il \textbf{calore scambiato} da un corpo, la \textbf{massa} del corpo stesso e la \textbf{variazione di temperatura}:
\[ 
Q = m \textcolor{red}{c}(T_f - T_i)
\]
dove \(\textcolor{red}{c}\) è il calore specifico caratteristico del corpo (osserviamo che il calore \(Q\) è una grandezza estensiva mentre il calore specivico \(c\) intensiva).

Poiché deve valere \(Q_1 = -Q_2\) vale
\[ 
m_1 c_1(T_e - T_1) = -m_2 c_2(T_e - T_2)
\]
dalla quale note 3 su 4 variabili si può ricavare la quarta. Nel caso in cui non si possa assumere che il calore specifico sia praticamente costante bisogna scrivere
\begin{equation}
	\viola{Q = \int dQ = m \int_{T_i}^{T_{f}}c(T)dT}
\end{equation}


\subsection{Calore specifico}
\begin{center}
	\fboxsep11pt
	\colorbox{yblue}{\begin{minipage}{5.75in}
			\begin{blues}{Calore specifico}
				Il calore specifico rappresenta il calore che occorre per scambiare con l'unità di massa di una data sostanza a temperatura \(T\) per farne variare la temperatura di \(\SI{1}{\kelvin}\).
				
				Il calore specifico è una grandezza intensiva caratteristica della sostanza.
			\end{blues}
	\end{minipage}}
\end{center}
\begin{center}
	\fboxsep11pt
	\colorbox{yblue}{\begin{minipage}{5.75in}
			\begin{blues}{Capacità termica}
				La capacità termica rappresenta il \textbf{calore da scambiare}  per farne variare la temperatura di \(\SI{1}{\kelvin}\).
				\[ 
				C = mc
				\]
			\end{blues}
	\end{minipage}}
\end{center}
Per variazioni infinitesime
\[ 
dQ  =mc dT \quad \to \quad c = \frac{1}{m}\frac{dQ}{dT}
\]
Spesso si preferisce far riferimento al calore scambiato daun certo numero di moli di sostanza, pertanto si definisce   anche il \textbf{calore specifico molare}
\[ 
c = \frac{1}{n}\frac{dQ}{dT}
\]
\begin{es}{precisazioni}
	E' però necessaria una precisaziione riguardo le definizioni di calore specifico. Quando la trasformazione avviene in assenza di lavoro scambiato con l'ambiente \(dW = 0\) e \(dQ = dU\) prt cui
	\begin{equation}
		c  = \frac{1}{m}\frac{dU}{dT} \qquad c  = \frac{1}{n}\frac{dU}{dT}
	\end{equation}
	equazioni che valgono solo quando il lavoro è nullo. Se però è compiuto lavoro esterno il calore scambiato dipende dalla trasformazione e è possibile definire diversi calori specifici per una stessa sostanza.
\end{es}
\subsubsection{Misura del calore specifico}
Si può effettuare la amisura del calore specifiico di un corpo tramite il \textbf{calorimetro di Regnault}. In un contenitore adiabatico si ha un recipiente pieno d'acqua a temperatura \(T_2\) con immerso un termometro e un agitatore. Si immerge quindi un corpo a temperatura \(T_1 > T_2\) e di calore specifico incognito \(c_x\). Raggiunto l'equilibrio a temperatura \(T_e\) il bilancio dei calori scambiati in modulo sarà
\[ 
m_xc_x(T_1 - T_e) = (C_1 + C_2)(T_e - T_2)
\]
dove il primo termine è il calore ceduto dal corpo e il secondo il calore assorbito dal calorimetro.
\[ 
c_x = \frac{(C_1 + C_2)(T_e - T_2)}{m_x(T_1 - T_e)}
\]
Naturalmente la misura non potrà essere più precida dell'1\% in quanto è impossibile ottenere un perfetto isolamento termico, inoltre la capacità termica del termometro deve essere adeguatamente piccola  in modo da minimizzare lo scambio di calore tra corpo e termometro.


\begin{center}
	\textcolor{red}{temp di Debye e legge di Dulong-Petit}
\end{center}

\subsubsection{Processi isotermi: cambiamenti di fase}
Si osserva che i cambiamenti di fase sono accompagnati da scambi di calore, più precisamente si tratta di quantità di calore \(\boldsymbol{\lambda}\) ben definite dette  \textbf{calori latenti}. Il calore richiesto per il cambiamento di fase da un corpo di massa \(m\) è 
\begin{equation}\label{calorelatente}
	\viola{	Q = m\lambda}
\end{equation}
La caratteristica importante dei cambiamenti di fase è di essere trasformazioni \textbf{praticamente reversibili}.

\begin{center}
	\fboxsep11pt
	\colorbox{attenzione}{\begin{minipage}{5.75in}
			\begin{attenzione}{!!!}
				Il calore latente \textbf{non ha un valore fisso}. Per esempio nel caso dell'evaporazione è funzione della temperatura.
			\end{attenzione}
	\end{minipage}}
\end{center}

\subsubsection{Sorgenti di calore}
\begin{center}
	\fboxsep11pt
	\colorbox{yblue}{\begin{minipage}{5.75in}
			\begin{blues}{Sorgente di calore}
				Si definisce \textbf{sorgente di calore} un corpo con capacità termica praticamente infinita e quindi con la proprietà di poter scambiare calore restando a temperatura costante
			\end{blues}
	\end{minipage}}
\end{center}
Grandi masse d'acqua o aria possono essere considerate sorgenti di calore; corpi meno massivi lo possono essere per tempi molto brevi.\\

\noindent
Proprio per la loro caratteristica di rimanere a temperatura costante, se un corpo è messo a contatto con una sorgente e la differenza di temperatura è finita, \textbf{non può esserci equilibrio termico} durante lo scambio. Può esserci l'equilibrio solo se si mantiene il corpo a contatto con la sorgente abbastanza a lungo (oppure se la differenza di temperatura durante lo scambio è infinitesima).


\section{Trasmissione del calore}

\subsection{Conduzione}
Prendiamo un corpo esteso in cui la temperatura non sia uniforme e tracciamo le \textbf{superfici isoterme} dove la funzione \(T(x,y,z) = \text{cost}\), diciamo che la superficie \(S_1\) ha temperatura \(T_1\) e così via. 

Se \(dS\) è un elemento della superficie isoterma, il calore che fluisce attraverso \(dS\) nel tempo \(dt\) è 
\begin{equation}\label{fourier calore}
	\viola{dQ = -\boldsymbol{k} \frac{dT}{dn}dS dt}
\end{equation}
\(\boldsymbol{k}\) è la \textbf{conduttività termica} del materiale e dipende dal materiale e dalla temperatura (dipendenza marcata nei metalli dove \(k\) cresce al decrescere della temperatura). Invece essendo \(\frac{dT}{dn}\) il \textbf{gradiente di temperatura} ortogonale a \(dS\) diretto verso temperature crescenti, il \(\boldsymbol{-}\) davanti a tutto indica che il flusso di calore è opposto al verso del gradiente di temperatura:\\

\noindent
definiamo \(J_U\) il \textbf{flusso di calore per unità di superficie} 
\[ 
J_U \equiv \frac{\Delta Q}{\Delta S \Delta t} = - k \frac{dT}{dn}
\]
che in forma vettoriale si può scrivere
\begin{equation}
	\viola{\overrightarrow{J}_U = -k \overrightarrow{\nabla}T}
\end{equation}

\noindent
Vediamo come usare in pratica la \ref{fourier calore}: consideriamo una parete piana indefinita \textbf{diatermica} di spessore \(s\) posta tra due ambienti a temperature \(T_1\) e \(T_2\). Andiamo ad analizzare una sezione della parete superficie \(dS\) in \(x\) attraverso la quale viene fornito il calore \(dQ_1\) alla massa \(dm\); intanto \(dm\) attraverso \(dS\) in \(x + dx\) cede il calore \(dQ_2\). Gli scambi di calore saranno quindi:
\[ 
dQ_1 = -k \left(\frac{\partial T}{\partial x}\right)_x dS dt
\]
\[ 
dQ_2 = -k \left(\frac{\partial T}{\partial x}\right)_{x+dx} dS dt 
\]
che tramite uno sviluppo di Taylor al primo membro
\[ 
\textcolor{gray}{f(x) = f(x_0) + f'(x)(x-x_0)}
\]
\begin{align*}
	\left(\frac{\partial T}{\partial x}\right)_{x+dx} =& \left(\frac{\partial T}{\partial (x+dx)}\right)_{x+dx} + \left(\frac{\partial^2 T}{\partial (x + dx)^2}\right)_{x+dx}(x - x +dx) \\
	=& \left(\frac{\partial T}{\partial x}\right)_x + \left(\frac{\partial^2 T}{\partial x^2}\right)_{x}dx
\end{align*}

\[ 
\to dQ_2 = -k\left[\left(\frac{\partial T}{\partial x}\right)_x + \left(\frac{\partial^2 T}{\partial x^2}\right)_{x}\right]dxdSdt
\]
Complessivamente \(dm\) assorbe il calore \(dQ_1\) da sinistra e cede \(dQ_2\) a destra, per cui assorbe 
\[ 
dQ_1 - dQ_2  = k\frac{\partial^2 T}{\partial x^2}_{x}dSdx dt
\]
quindi ricordando che per variazioni infinitesime vale
\[ 
dQ = dmcdT = \rho dS dx c dT
\]
troviamo che 
\[ 
k\frac{\partial^2 T}{\partial x^2}_{x}\textcolor{red}{dSdx} dt = \rho  \textcolor{red}{dSdx} c dT
\]
ottenendo la legge che regola la variazione di temperatura in funzione del tempo e della posizione della parete (con \(T(x=0) = T_1\) e \(T(x = s) = T_2\))
\begin{equation}
	\viola{\frac{\partial^2 T}{\partial x^2} = \rho \frac{c}{k} \frac{\partial T}{\partial t}}
\end{equation}
\begin{es}{A regime}
	Arrivati alla temperatura di regime e una volta che la temperatura di ciascun punto ha raggiunto un valore stazionario \textbf{costante nel tempo} si annulla la derivata temporale e di conseguenza anche la derivata seconda spaziale:
	\[ 
	\frac{\partial T}{\partial t} = 0 \quad \Rightarrow \quad \frac{\partial^2 T}{\partial x^2} = 0\quad \Rightarrow \quad T(x) = \alpha + \beta x
	\]
	imponendo
	\begin{align*}
		&T(0) = T_1 = \alpha  \\
		&T(s) = T_2 = \alpha + \beta s = T_1 + \beta s
	\end{align*}
	\[ 
	\beta = \frac{T_2 - T_1}{s} 
	\]
	possiamo riscrivere l'andamento della temperatura come
	\[ 
	T(x) = T_1 - \frac{T_1 - T_2}{s}x \qquad |\overrightarrow{\nabla}T |= \frac{T_1 - T_2}{s}
	\]
	che mostra come la temperatura decresca linearmente partendo dal valore massimo (\(x=0\)) \(T_1\) fino al valore minimo (\(x=s\)) \(T_2\). \\
	
	Prima avevamo definito il il calore che flusice attraverso una superficie \(dS\) in un tempo \(dt\) come
	\[ 
	dQ = -k\frac{dT}{dn}dS dt
	\]
	dalla quale ricaviamo che, nel nostro caso, attraverso \(dS\) flusice il calore 
	\[ 
	Q = k\frac{T_1 - T_2}{s}St
	\]
	\begin{center}
		\fboxsep11pt
		\colorbox{attenzione}{\begin{minipage}{5in}
				\begin{attenzione}{!!!}
					In \textbf{regime stazionario} il calore che entra nella parete dal lato ad alta temperatura è uguale al calore che esce dal lato a assa temperatura. \textbf{C'è solo flusso di energia} e nessuna cessione alla parete.
				\end{attenzione}
		\end{minipage}}
	\end{center}
\end{es}

\paragraph{Equazione di continuità} Presa una superficie orientata \(\Sigma\) in un mezzo non in equilibrio termico di volume \(V\), in assenza di sorgenti di calore in \(V\) il calore che esce per unità di tempo da \(V\) attraverso \(\Sigma\) è uguale a meno la variazione per unità di tempo del calore contenuto in \(V\):

\[ 
\int_{\Sigma} \overrightarrow{J}_U \cdot \hat{n} dS = -\frac{d}{dt}\int_V \boldsymbol{q} dV = - \int_V \frac{\partial q}{\partial t} dV
\]
ocn \(\boldsymbol{q}\) energia termica per unità di volume. \\

\noindent
Se andiamo a riscriverla in forma differenziale otteniamo
\[ 
\overrightarrow{J}_U \cdot \hat{n}dS = - \frac{\partial q}{\partial t} dV
\]
e quindi come per i fluidi ()
\begin{equation}
	\viola{	\frac{\partial q}{\partial t} + \overrightarrow{\nabla} \cdot \overrightarrow{J}_U  = 0 }
\end{equation}

\begin{center}
	\textcolor{red}{inserire grafici}
\end{center}


\subsection{Convezione}
La conduzione termica nei fluidi è difficile da osservare perché in essi avviene un altro fenomeno di trasmissione del calore: la convezione.

Riscaldando un fluido la parte più vicina alla sorgente di calore si scalda prima dilatandosi. Le parti di fluido più calde risentono quindi di una spinta di Archimede maggiore e tendono a spostarsi verso l'alto generando correnti ascensionali che fanno avvicinare le parti di fluido più fredde alla sorgente. 

\subsection{Irraggiamento}
L'energia emessa sotto forma di onde elettromagnetiche in unità di tempo è descritta dalla \textbf{legge di Stefan-Boltzman} come
\begin{equation}\label{stefan boltzman}
	\viola{\varepsilon = \boldsymbol{\sigma e}T^4}
\end{equation}
dove \(\boldsymbol{\sigma} = 5.67\cdot 10^{-8}\SI{}{\frac{\joule}{\second\meter^2\kelvin^4}}\) è una costante universale ed \(\boldsymbol{e}\) è l'\textbf{emissività} che varia tra \(0\) e \(1\): \(0\) per le pareti riflettenti, \(1\) per le cosiddette superfici nere. Una superficie nera assorbe tutta l'energia che incide su di essa.

\subsection{Calore tra solido e fluido}
La trasmissione del calore tra un corpo solido a temperatura \(T\) e un fluido a temperatura \(T_0 < T\) messi a contatto tramite una superficie \(S\) è descritta dalla \textbf{legge di Newton}
\begin{equation}
	\viola{Q = \boldsymbol{h}(T-T_0)St}
\end{equation}
dove \(\boldsymbol{h}\) è la \textbf{conducibilità termica esterna}

\newpage
\chapter{Gas ideali e reali}
\begin{es}{}
	\begin{enumerate}
		\item Equazione di stato dei gas ideali;
		\item Trasformazioni di un gas ideale;
		\item Cicli termodinamici di un gas ideale;
		\item Teoria cinetica dei gas ideali
		\item Gas reali;
	\end{enumerate}
\end{es}


\begin{center}
	\fboxsep11pt
	\colorbox{yred}{\begin{minipage}{5.75in}
			\begin{redes}{}
				\subsection{Legge isoterma di Boyle}
				Per un gas ideale in equilibrio termodinamico (\(p_{0}, V_{0}, T\)) il prodotto \(pV\) è costante  in tutti gli stati di equilibrio collegati da \textbf{trasformazioni isoterme e non}.
				\[ 
				\boldsymbol{pV = \text{ cost}}
				\]
			\end{redes}
	\end{minipage}}
\end{center}

\begin{center}
	\fboxsep11pt
	\colorbox{yred}{\begin{minipage}{5.75in}
			\begin{redes}{}
				\subsection{Legge isobara di Gay-Lussac}
				Per un gas ideale in equilibrio termodinamico (\(p_{0}, V_{0}, T\)) vale
				\[ 
				\boldsymbol{V = V_0(1 + \alpha t)}
				\]
				in tutti gli stati di equilibrio collegati da \textbf{trasformazioni isobare}.
			\end{redes}
	\end{minipage}}
\end{center}

\begin{center}
	\fboxsep11pt
	\colorbox{yred}{\begin{minipage}{5.75in}
			\begin{redes}{}
				\subsection{Legge isocora di Gay-Lussac}
				Per un gas ideale in equilibrio termodinamico (\(p_{0}, V_{0}, T\)) vale
				\[ 
				\boldsymbol{p = p_0(1 + \beta t)}
				\]
				in tutti gli stati di equilibrio collegati da \textbf{trasformazioni isocore}.
				
				\[ 
				\alpha = \beta = \frac{1}{273.15} ^\circ C^{-1} \qquad \text{per un gas ideale}
				\]
			\end{redes}
	\end{minipage}}
\end{center}


\subsection{Legge di avogadro}
Volumi di gas diversi, aventi stessa temperatura e stessa pressione, contengono lo stesso numero di molecole.\\
Il volume di un gas dipende quindi dal numero di molecole da cui è costituito e non dal tipo di molecole o dalla loro dimensione. \textit{A stessa temperatura e pressione un litro di ossigeno contiene lo stesso numero di molecole di un litro di elio.}

Si chiama mole una quantità di materia che contiene tante entità elementari quanti sono gli atomi contenuti in 0.012 kg dell'isotopo $^{12}C$ del carbonio, ovvero $N_A = 6.022\cdot10^{23}$ entità elementari. 

\subsection{Equazione di stato dei gas ideali}
Consideriamo n moli di gas reale alla pressione atmosferica $p_0$, temperatura $T_0 = 274.15\,K$ e volume $V_0 = nV_m$.
Mantenendo costante il volume e portando la temperatura al valore T abbiamo da Gay-Lussac che: 

\[ 
p_T = p_0 \alpha T \quad \Longrightarrow \quad p_T V_0 = p_0 (V_0 \,\alpha\, T) = p_0 V_T
\]
Nel secondo passaggio si è moltiplicato per $V_0$ ambo i lati dell'equazione.\\ $V_0$ e $p_T$ sono le coordinate termodinamiche in un particolare stato di equilibrio alla temperatura T, come lo sono $p_0$ e $V_T$ sempre alla temperatura T. In base alla legge di Boyle si ha: 
\[ 
p_T V_0  = p_0 V_T = pV
\]
essendo p e V le generiche coordinate di uno stato di equilibrio alla temperatura T. Otteniamo dunque:
\[ 
pV \;=\; p_0 \,V_0\, \alpha\, T \;=\; n\, p_0\, V_m\, \alpha\, T \;=\; n\, R\, T  
\]
dove il prodotto $p_0\, V_m\, \alpha\, = R$  è una costante universale per tutti i gas, ovvero $R = 8.314\, J/mol\,K$. Dunque abbiamo ricavato l'equazione di stato dei gas ideali: 
\begin{equation}
	\viola{pV = n\,R\,T }
\end{equation}

\section{Trasformazioni  di un gas}
Una trasformazione che porti un gas da uno stato A ad uno stato B può svolgersi attraverso stati di equilibrio termodinamico ed è rappresentabile nel piano di Clapeyron da una curva continua. Se invece la trasformazione avviene attraverso stati di non equilibrio si una rappresentazione a tratti per indicare che si ignorano i valori delle coordinate termodinamiche durante il processo.



utile però solo se si conosce l'espressione esplicita di p(V), situazione soddisfatta sostanzialmente in due sole situazioni: 
\begin{itemize}
	\item[a)] la \textbf{trasformazione è reversibile} e pertanto si può calcolare l'integrale in quanto la pressione è determinata in ogni stato intermedio; 
	\item[b)] \textbf{è nota la pressione esterna} che, per esempio, è costante; caso tipico di quando il processo avviene sotto pressione atmosferica; allora, anche se la trasformazione non è reversibile, il lavoro è calcolabile ed è dato da  $\;W = p_{amb}\,(V_B - V_A)$
\end{itemize}
In tutti gli altri casi in cui la pressione non è nota la pressione non si può applicare la (\ref{lavoro}). 
Ad ogni modo se la trasformazione è \textbf{isocora} il lavoro è nullo in quanto $V$ = costante. 


Il lavoro, se si può utilizzare (\ref{lavoro}), in accordo con il significato geometrico dell'operazione di integrazione è pari all'area compresa tra la curva e l'asse dei volumi. In una trasformazione ciclica reversibile il lavoro è dato dall'area racchiusa dal ciclo stesso.

\subsection{Calore e energia interna del gas ideale}
In una trasformazione generica un gas scambia anche calore con l'ambiente e per il calcolo bisogna ricorrere al primo principio. Per una trasformazione infinitesima isocora si ha $dQ = n\,c_V\,dT$. \\Le quantità:
\[ 
c_V = \frac{1}{n}\left(\frac{dQ}{dT}\right)_V \quad,\quad c_p = \frac{1}{n}\left(\frac{dQ}{dT}\right)_p
\]
si chiamano, rispettivamente, calore specifico molare a volume costante e a pressione costante e si misurano in $J/\text{mol}\,K$. 
Se $c_V$ e $c_P$ possono essere ritenuti costanti, il calore scambiato per una variazione $\Delta T$ di temperatura risulta: 
\begin{equation} 
	Q_V = n\,c_V\Delta T \quad,\quad Q_p = n\,c_p\Delta T
\end{equation}
Altrimenti
\begin{equation} 
	Q_V = n\int_{A}^{B} c_V\,dT \quad,\quad Q_p = n\int_{A}^{B} c_p\,dT
\end{equation}
Come nel caso dei solidi e dei liquidi il calore scambiato dipende solo dalla variazione di temperatura. In un gas ideale risulta inoltre sempre vero $c_p > c_V$. 



\section{Relazione di Mayer}
th
\section{Trasformazioni adiabatiche}
Il gas è racchiuso da pareti adiabatiche e può scambiare lavoro solo con l'ambiente (tramite una parete mobile).

\begin{align*}
	Q = 0 \quad \Rightarrow \quad W_{AB} =& -\Delta U \\
	=& -nc_{V} \Delta T
\end{align*}

\[ 
pV = nRT \quad \Rightarrow \quad nT = \frac{pV}{R}
\]
Presa una trasformazione dal punto \(A\) al punto \(B\) si avrà quindi
\begin{align*}
	W_{AB} =& nc_{V}(T_{A} - T_{B}) = c_{v}\frac{(p_{A}V_{A} - p_{b}V_{B})}{R} \\
	=& \frac{1}{\gamma -1}(p_{A}V_{A} - p_{b}V_{B})
\end{align*}
con 
\[ 
\boxed{c_{P} - c_{V} = R}
\]

\section{Trasformazioni}
\section{Rendimento}
\section{Cicli}
\section{Gas reali}
\section{Teoria cinetica dei gas ideali}



\chapter{Termodinamica - II principio}
Sperimentalmente si ottiene che è possibile trasformare integramlmente il lavoro in calore ceduto ad una sorgente lasciando il sistema nello stato iniziale, ma non è possibile trasformare integralmente il calore prelevato dalla sorgente in lavoro.

\begin{center}
	\fboxsep11pt
	\colorbox{yred}{\begin{minipage}{5.75in}
			\begin{redes}{}
				\subsection{Enunciato di Kelvin-Plank}
				È impossibile realizzare un processo che abbia come unico risultato la trasformazione in lavoro del calore fornito da una sorgente a temperatura uniforme.
				\vspace{0.2cm}
				\subsection{Enunciato di Clausius}
				È impossibile realizzare un processo che abbia come unico risultato il trasferimento di una quantità di calore da un corpo ad un altro a temperatura maggiore.
			\end{redes}
	\end{minipage}}
\end{center}

Con l'aiuto di una particolare sistema di macchine termiche si può facilmente dimostrare l'equivalenza dei due enunciati: \\

\noindent
Prendiamo due macchine termiche che lavorano tra due sorgenti \(T_{1}\)
 e \(T_{2}\).
 \begin{enumerate}
 	\item Supponiamo non valga il principio di Kelvin-Plank e che quindi sia possibile realizzare un processo ciclico  che trasformi integralmente calore in lavoro (è nulla la cessione di calore alla sorgente fredda). Il lavoro prodotto è utilizzato per far funzionare una macchina frigorifera che assorbe il calore \(Q_{1}\) dalla sorgente fredda \(T_{1}\) e cede calore \(Q_{2}\) alla sorgente \(T_{2}\) calda.
 	
 	\textbf{Si ha quindi conme risultato il trasferimento di calore da una sorgente fredda a una calda violando l'enunciato di Clausius.}
 	
 	\item Immaginiamo una prima macchina termica che viola l'enunciato di Clausius che scambia il calore \(Q\) da \(T_{1}\) a \(T_{2}\) \(Q\). Se dimensioniamo una seconda macchina in modo che questa assorba \(Q_{2}\) da \(T_{2}\), compia il lavoro \(W\) e ceda \(Q\) a \(T_{1}\). In questo modo la macchina complessiva non scambia calore con \(T_{1}\) e compie il lavoro \(W = Q_{2} + Q\) con \(Q_{2}\) assorbito  da \(T_{2}\) e \(Q\) ceduto a \(T_{2}\). \\  
 	
 	\textbf{Il risultato è una macchina che trasforma tutto il calore scambiato con la sorgente \(T_{2}\) (\(Q\) e \(Q_{2}\)) in lavoro, violando quindi l'enunciato di Kelvin-Plank.}
 \end{enumerate}
 
 \subsubsection{Ciclo monotermo}
 Il primo enunciato evidenzia come non sia possibile effettuare un processo ciclico con una sola sorgente. L'unica situazione possibile è quella con 
 \[ 
 Q \leq 0 \qquad W \leq 0
 \]
 dove viene compiuto lavoro sulla macchina che cede calore alla sorgente. Se la macchina è \textit{reversibile} allora il processo può essere percorso al contrario 

\end{document} 

undefined